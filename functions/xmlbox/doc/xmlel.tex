\documentstyle{article}
\begin{document}

\title{xmlel——xml元素类设计}
\author{lymslive@403708621}
\date{2015-1}

\section{前言简介}

xml 是通用数据文件格式,程序处理 xml 常用一种 dom 模型。matlab 也可直接使用
dom,它是一个 java 对象。但我使用时常觉有些不便,尤其是 xmlread 解析 xml 后,
元素属性被排序,不保持 xml 源文件中的顺序。xml 规范是属性顺序无关,但我的工作
中,期望保持属性顺序,以便查看与编辑。
\footnote{新版matlab 的 xmlread 不知是否会保持属性顺序}

另外,dom 的模型实现是隐藏的,我希望自己有更多的控制。因此,我根据自己对 xml
的理解,用 matlab 的类来实现 xml 的数据模型。这个解决方案不求完美、完美,或许
不能通用地处理任何 xml ,但期望它能以 matlab 的程序习惯,处理绝大多数 xml,尤
其是作为数据文件的 xml。

这处 matlab 的 xml 解决方案,主要用于处理配置文件,提取配置文件的数值数据,利
用 matlab 强大的数学计算分析能力进行处理,修改 xml 然后原样保存。同时也能将
matlab 的常规基本数据类型,如矩阵、元胞、结构等,转化为 xml ,以便形象化地透视
matlab 的变量结构,并且可能方便地用外部工具修改 xml 表示的 matlab 变量。

但这不是要替代 matlab 保存变量至 .mat 格式的做法,也不能完全替代 dom 与 xpath
等通用的 xml 解析方案。只是多提供一种强力的基础工具,以便简化 matlab 操作 xml
的方法。

此前也参考了其他 matlab 写的 xml 解决方案,多有借鉴意义,却也不完全满意,故而
已想自己实现一套方案。

\subsection{xmlel 抽象模型}

xmlel 致力于用 matlab 类对象抽象表达 xml 的元素结点。一个 xml 元素具有标签名、
属性、子元素这三个重要结构属性。

文本当作一种特殊元素,它还有个域用以储存这个文本,但它不再有子元素结点。常规元
素没有文本属性,它只能包含一个文本元素结点。文本元素结点也没有属性。

在通用的 dom 模型中,属性也当作一种结点,但在 xmlel 类中,元素属性使用简单的
struct 结构,一个元素只有唯一的 struct,但这个 struct 可以有多个域。属性值保存
字符串,由外部调用者决定是否需要转化为数字。

为表达 xml 的层次关系,每个 xmlel 对象还有一个父指针 parent,即另一个 xmlel 对
象实例。因为 xmlel 继承自 handle 类,因此相当于指针。但 xmlel 对象不额外保存兄
弟前驱或后驱指针。按 matlab 的习惯,使用 xmlel 对象数组,按数组位置来表达同层
次元素对象的兄弟关系。一般地,用对象的列向量。每个 xmlel 的子元素 child 就是一
个 xmlel 列向量,即其他元素的指针。

xmlel 对象标量可视为一个 xml 文档,因为它是单根的。但 xmlel 数组并不是有效的
xml 文档,因为它是多根的,但对象数组可视为 xml 文档片断。xml 文档的根元素是没
有父元素的。xmlel 对象初始化时 parent 指针也为空。为区分根元素与“悬”元素,约
定将作为 xml 文档的根元素的父指针设为 true(1),其他未确定父元素的元素 parent
指针留空 [] 或设为 false(0)。

\subsection{xml 源文件良构性}

在 xml 的实际运用中,xml 文件由一些专门工具或编辑器生成的话,除了符合 xml 的语
法规范,还具有良好的格式。具体地,若源文件具有以下特征,称之为良构:

1. 每行只有一个元素
2. 元素的所有属性写在开标签的同一行
3. 自封闭的元素占一行,如果元素只有简短的文本内容,开标签与闭标签也在同一行
4. 子元素从下一行开始,并缩进
5. 在所有子元素结束后,闭标签另外单独一行

这种格式规范是很自然的。本解决方案在解析源文件时依赖于这种格式假设,以行为单位
扫描源文件,可以较方便地用正则表达式解析。当然程序对象表达的 xml 元素是与源文
件无关的。将 xmlel 对象格式化为 xml 文件保存时也遵循以上良构规范输出。另外须指
出,源文件的缩进只为人工易读,对程序解析并无影响,但务必不要违反行原则。

此外,以保存数据为主要目的的 xml 文件,还有一些潜规则或习惯。即数据主要保存在
元素的属性值中,而不是保存在元素文本中。而且,同名子元素一般是连续集中的。以格
式化文档为目的的 xml 却不是这样,文本元素与表示格式的元素交叉使用。

matlab 显然只适于专注处理数据的 xml 文件。不过 xmlel 对象的 child 数组允许同名
元素分散。

\subsection{平易型的 xml 结构化}

不专门建立类对象,也能用 matlab 的 struct 保存一个 xml 的结构数据。这里定义一
种 struct 规范来表示 xml,称为 plain 型的 struct:

1. 直接用每个子元素的名称,当作父元素的一个域名,子元素是另一个嵌套 struct;
2. 在子元素外,另有一个特殊的属性域,它是另一个结构,保存元素的所有属性;
3. 再有一个文本域,保存该元素的文本字符串;
4. 多个同名子元素,保存于父元素的同一个域名下,以 struct 数组的形式保存,但是
如果它们各有不同的子元素,则不能用 struct 数组,而改用 struct cell,即元胞数组
的每个元胞内容保存一个同名子元素。

这种 struct 结构的一个方便之处是可以连续使用点号取域名索引子元素。但是遇到一组
同名元素时会麻烦点。它更适用于层次较简单的场合,尤其是多个同名元素只在最底层叶
子。另一个不便之处是 struct 按值传递参数,要用函数修改结构时花销较大。

曾在网上找到其他外部工具能将 xml 转化为这种 struct 结构。我初始也想用这种结构
来处理 xml ,但最终还是将子元素封装为对象数组来处理,接口会更一致。点号索引子
元素可以重载来实现,只是无法在命令窗口补全而已。

我这里提出的 xml 解决方案,也会考虑与这种 plain struct 的转化。不过在将 xmlel
对象保存(saveobj)时,还是采用类似 xmlel 的结构化,即将子元素封装在结构数组中,
标签名保存在结构本身的一个域值中,而不是充当父元素的域名。

\subsection{matlab 变量的 xml 化}

将 matlab 的常用数据类型转化为 xml ,遵循以下规则:

1. 矩阵,指普通的数值矩阵。将每一行视为一个 xml 元素,每一列视为元素的一个属性
,属性名称用字母列号,即 A, B, \ldots , Z 等。当然转化方法也允许接收定制的属性
名称。不过逆向提取矩阵时,可能失去列名属性的信息,除非也使用有列名的封装矩阵类
,比如我另外写的 incTable 类。不考虑数值字符的精度问题,而且整数矩阵也不会有精
度损失。最终的结果是,一个矩阵将转化为一个 xmlel 对象列向量,对象数组的长度与
原矩阵的行数相等。也不考虑三维以上的矩阵。

2. 元胞。cell 数组分两类。第一类是相当于矩阵的简单元胞,称之为标量元胞,即每个
元胞的内容都是标量数值或字符串,仅仅是因为有数值与字符串混合,而不能普通矩阵,
只能用元胞矩阵保存数据。对于这种简单元胞矩阵,可以像矩阵一样转化为 xml ,因为
数值在 xml 中也是要转化为数字字符串的。所以,转化结果同样是,每行一个元素,每
列一个属性。

第二类元胞数组,元胞内容不全是标量,而是更复杂的数据,由此不能将该元胞的值放在
某个 xml 元素的属性上。对于这类元胞数组,将每个元胞转化为一个 xmlel 对象,其默
认的元素名就用 'cell',以表示这是一个元胞数组,然后将元胞的值转化为该元素的子
元素。结果生成的 xmlel 对象数组,其个数是元胞的个数即行数乘列数——事实上也支
持多维元胞矩阵。

3. 结构。struct 与 xml 是最相近的。每个 struct 转化为一个 xml 元素,元素名默认
使用'struct'。然后每个域转化为元素的子元素。如果是 struct 数组,则转化为相同个
数的 xmlel 对象数组,列向量。

第二类元胞与 struct 转为 xml 都至少有两个层次,第一个层次一般是没有属性的。或
者可以考虑添加类似 matalb:Class='cell' 之类的助于反向解析的属性。另外,需要递
归处理转化元胞值或结构域值。

4. 字符串。短字符当作一个标量,放在元素属性上。长字符放在元素文本中。多行 char
数组转化为 xmlel 对象数组。cellstr 类似普通矩阵转化。matlab 不期望处理大量文本
。

5. 逻辑值。logical 标量可当作标量整数 1/0 或标量字符串 true/false 处理。输出为
xml 时转为 true/false,解析读入时,转化为 1/0 ,便于使用矩阵。这虽然会导致不可
逆,但逻辑值不是太常用。有必要时再特殊处理。

6. 其他类型。其他复杂类型大多可用 struct 或 cell 数组表示,可以递归转化为 xml
。自定义对象,无法转为 xml ,除非提供相应方法。

\section{使用方法}

解析 xml 源文件,用 xmlfile 类,它有个 parseFile 方法,解析后的根元素存在
xroot 属性中:

fx = xmlfile(filename);
fx.parseFile('xmlel');
xml = fx.xroot;

将一个 matlab 变量转为 xmlel 对象,用 xmlel.cast(var) 方法。

将 xmlel 对象转为 xml 格式文本,用 char 方法。写入文件用 write 方法。

将 xmlel 对象(数组)转为 matlab 变量,用 tomat方法,能确定类型的话,也可直接
用 double struct cell 转化方法。

增删子元素用 pushChild 与 popChild。遍历兄弟元素可用 nextSibling prevSibling。
用 rootas 将自己设为根元素。

直接构造 xmlel 对象用 xmlel(name, count) 创建 count 个标鉴名为 name 的元素对象
,可只指定其中一个参数。一个复杂 matlab 变量也可传入构造函数中,将会调用 cast
方法转化。

命令窗无分号结束,可打印查看标量对象,或统计对象数组信息。

\section{程序实现}

\subsection{xmlel 元素对象定义}

xmlel < handle
基础属性:name, at, child,读取公有,赋值用 set. 方法保障
指针属性:parent,透明的属性,不必保存的属性
常量属性:RealRoot, NoneRoot
非独属性:atname, vector

在常规元素中,name 是元素名或标签名,标量字符串。at 是标量 struct,每个域保
存着元素的属性。child 是子元素,xmlel 对象数组,列向量。

文本元素特殊处理,name 的值设为 '#Text',而将字符串保存在 child 属性中。这是为
了节省储存空间,文本元素不会用子元素,利用 matlab 弱类型的特点,在这种情况下将
字符串赋与 child。但外界接口无需关注这个特殊性,提供了接口方法 isTextNode 来判
断是否文本元素,以及 getText 来取出文本元素的字符串。

其实,属性也可以用 xmlel 对象来表示,属性名以 @ 前缀来区分。不过用 struct 基本
类型来保存,会节省与简化许多。

为维护父指针,设定两个特殊常量属性。RealRoot=true, NoneRoot=false。确实是 xml
根元素的父指针赋与 RealRoot,其他非根元素而未定父元素的悬元素,父指针留空或指
向 NoneRoot。matlab 是无类型的,可以将任何值赋给一个变量。所以 parent 允许设值
为这两个逻辑值,其他情况在 set. 方法中保证指向另一个有效的 xmlel 元素。

atnmae vector 分别表示属性名与属性值,只可由标量获取。

\subsection{xmlel 构造方法}

obj = xmlel(varargin) 构造方法设计为允许 0 至 2 两个参数。无参数时构造空对象。
一个参数时,允许标量字符串,用之作为标鉴名创建一个元素对象;或者一个标量整数,
表示创建对象数组,每个对象都是未经初始化的空对象。两个参数时,分别是标量字符串
,与标量整数,创建有相同名字的元素对象数组。

构造函数会调用 name 属性的 set 方法,检查合法的名。对象的其他属性,不在构造
函数赋值,一般是在用名称创建合法元素后外部赋值,依然会用 set 方法保障合理性。

\subsection{转化方法:将 matlab 变量转化为 xmlel 对象}

静态方法 cast 将其他 matlab 变量转为 xmlel 对象(数组)。在构造函数中,如果单
参数不是标量字符串(元素名)或标量整数(对象个数),则调整该静态方法,尝试将这
个变量转化为 xmlel 对象。

cast 方法支持常规的矩阵、struct、cell,这即涵盖了 matlab 中能用到的绝大部分数
数据类型。其他自定义类型如果实现的转化方法 xmlel,也会调用之。

obj = cast(var, name, atname)

1. 矩阵包括 double 数值矩阵与简单 cell 矩阵,后都指的是每个 cell 的内容都是一
个标量数字或字符串。由于转化的元素属性的值都以字符串表示,故 double 矩阵在实现
时也是先将为简单 cell 字符串矩阵。

矩阵转化将返回 xmlel 对象列向量,每一行被转为一个元素,每一列被转为一个属性。
这样的 xml 在一些 xml 文件编辑器中可栅格化为表格,类似 matlab 矩阵变量的模型。
也因此要求必须是简单矩阵。属性名即列名取自字母序 ABC \ldots Z,超过 26 用 c27
格式表示。

每一行的元素名默认用 'row' 表示,但也可接收额外参数,指定元素名。例如从数值笼
矩阵转化的对象,元素名会默认设为 'double'。属性名也可传入参数代替默认的字母列
名。元素名是标量字符串,属性名是 cellstr。

2. 复杂 cell 数组(向量或矩阵),每一个 cell 转化为一个 <cell> 元素,当然
'cell' 是默认元素名,可自定义元素名。而将 cell 的内容值,递归转化为它的子元素
。

struct 变量,每个 struct 转化为一个 <struct> (或自定义元素名),然后将它的每
个域递归转化为其子元素,且将域名作为相应子元素的元素名。若是 struct 数组,则将
得到 xmlel 数组。

不过 cell 数组或 struct 数组的形状如何,返回的 xmlel 数组都是列向量。类似 xml
文件中的元素垂直排列。只用到 cast 方法传入的 name 参数自定义元素名,递归的子元
素只能用默认的元素名与属性名。且 <cell> 或 <struct> 父元素不再需要额外属性,因
为数据只放在子元素的属性中,本身的属性无值。

由于 <cell> 与 <struct> 的元素名可自定义,故无法从元素名来区分源变量是 cell 或
struct。内在区别是 cell 内只有一个值,即一种元素名,而 struct 内可以有多个域值
,即多种元素名。如果 struct 只有一个域,那两者被转化后也无从区别;但是只有一个
域的 struct 变量,建议用 cell 储存更合适。

3. 字符串变量。短字符串转为 cellstr ,然后转化元素后,保存在属性值上。长字符串
,认为不适合保存在属性值上,则转化为文本元素,并建立一个包含它的父元素。

4. 其他,逻辑值将转化为小写的 'true'/'false' ,当作标量保存在元素的属性值上。

\subsection{转化方法:将 xmlel 对象转为 matlab 变量}

1. char() 将 xmlel 对象表示的 xml 元素格式化为 xml 文本字符串。该字符串可能很
长,如果是单对象,它就是规范的 xml 文档,可保存至物理文件中。转化的 xml 字符串
或文件,遵循上述的良构规范。

2. double() 将 xmlel 对象数组,转化为矩阵,提取元素的属性值构成矩阵。如果都是
可转为数值的话,就返回数值矩阵,否则返回元胞矩阵。

3. struct() 有几种方式将 xmlel 对象转为 struct,需要提供额外参数。默认行为尽量
逆解析为从 struct 普通变量转化的 xml。

4. cell() 将 xmlel 转为 cell ,也至少有两种不同需求。一种是逆解析 cell 变量转
的 xml 。另一种是也可用 cell 来表示 xml 的层次结构。默认使用前者。

5. tomat() 是更通用的转化接口方法。当不确定元素对象(数组)表示矩阵或 struct
还是 cell 时,tomat()将自动识别转化。优先级是先识别属性值,将其转化矩阵,如果
全是数字字符串,还转为 double 矩阵,否则 cellstr 矩阵。没有属性值时,将子元素
转为 cell (单元素名)或 struct (多元素名)。

6. atMatrix 是更内部基础的方法,用于将同名元素的属性值转化为矩阵。tomat 等方法
都会调用 atMatrix。

\subsection{定制行为}

1. display

在命令窗显示时,标量对象打印格式化 xml 文本。对象数组时,统计各种元素名标鉴的
个数。

2. 保存对象

保存对象不必做特殊处理,只要把 parent 指针标为透明属性(Transient),就会递归保
保存子元素,但不会循环保存父元素。在给子元素即 child 属性赋值时,维护子元素的
父指针,即将子元素的 parent 属性指向自己。这样在用 load 加载对象变量时,能得到
正确指针关系的元素对象树,但顶层对象的父指针仍为空。xmlel 对象表示 xml 元素结
点对象,不是 xml 文件,故允许父指针为空。可在加载后需要时将元素设为根元素。

3. 重载索引

由于要处理 xmlel 对象数组,这部分处理会更复杂。需要区别对象标量与向量作不同处
理。

display 打印 xmlel 变量拥有多少个元素,按元素名分组统计。如果是标量,显示其格
式化文本,即 char() 的输出,如果输出过多,只预览前几行。

由于要处理数组,小括号是最好不要重载了。点号索引可重载,实现类似 xpath 的功能
。

\subsection{指针维护}

xmlel 对象通过包含 child 数组,建立 xml 元素树结构。但由于反向的父指针是不透明
属性,故需要另外处理维护。

parentas 方法将自己作为另一组对象的父元素,若不提供额外参数,则递归地将子孙元
素建立正确实的父元素指针。

parentto 方法将自己的父元素指针指向其他一个元素对象,若不提供额外参数,则将自
自己的父指针置否(类常 NoneRoot)。

parentas 与 parentto 方法,都要求作为父元素的一方为标量对象(或特殊的逻辑值)
,作为子元素的一方可以是对象数组。但这两个方法只修改 parent 属性,不修改 child
属性值,故不能保证 parent 元素的 child 数组包含自己。修改 child 属性见下一节的
修改方法。

这两个方法修改指针比较危险,定为私有方法。但开放两个相似的方法。

rootas 方法将自己设为根元素,可以接收额外参数,将另一组对象添加到自己的子元素
中。经此方法后,自己就可成为一个有效的 xml 文档模型。当然只能由标量对象调用该
方法。

rootto 方法为自己(可以是对象数组)创建一个根元素。接收一个字符串标量,用于新
建根元素的名称,默认用 'root' 。也可以直接接收另一个元素对象,将自己添加到它的
子元素中,并将它设为根元素。

一般不必关心这几个方法返回值。parentas(to) 返回成功建立父指针的个数,
rootto(as) 返回状态参数,操作成功返回 0,失败返回 1。

\subsection{修改方法}

对标量元素对象的修改,直接用赋值方法。其中 name 属性与 at 属性都是基本 matlab
变量,值传递参数。set 方法检查值的有效性。

child 属性是另一组元素对象,句柄对象。赋值时除了检查有效性,还构建正确的父子指
针关系。

对自身的对象数组操作,可直接用 matlab 的数组方法。另外增加几个方法便于操作子元
素数组(自己必须是标量)

pushChild 方法用于将其他一些元素对象添加到自己的子元素列表末尾,或在指定的位置
插入。该方法会调用 parentas 同时维护新增子元素的父指针。

popChild 方法用于删除一个子元素结点,若不提供位置序号,则删除最后一个。

deleteChild 方法用于按元素名删除一些元素结点,若省略参数,则清空所有子元素。

indexChild 方法用于取得自己在父元素的子元素列表中的位置,失败返回 0。

nextSibling 方法用于取得自己的下一个兄弟元素,失败返回空。

prevSibling 方法用于取得自己的上一个兄弟元素,失败返回空。

注意,对元素列表的操作都是针对子元素列表的,如果真想对自身列表操作而又不想用
matlab 原生的数组方法,可先将自己 rootto 到一个根元素,然后对其进行操作。

\subsection{特殊元素处理}

处理文本元素的方法有:
createTextNode :静态方法,创建一个文本元素
isTextNode :判断一个元素是否文本元素
getText :取得文本元素的字符串

注意:文本元素的名称为 '#Text' ,字符串则保存在 child 属性中。
xmlel.pushChild
isEmptyNode 判断是否为空元素,没有元素名称的元素即判为空元素,一般只在空参调用
构造方法时才会出现空元素。其他场合很少遇到,也不能为 name 属性赋空值。

isEmptyNode 与 isTextNode 都可由对象数组调用。但另一个 isroot 只能由标量对象调
用,用以判断一个对象是否为根元素,只能有一个根元素。

count 方法用于统计对象数组中,不同名称的元素各有多少个。由于在实现中用 struct
的办法来当作“伪”hash 来使用,故把空元素与文本元素的名称改用 'NONE' 与 'TEXT'
代替。一般情况下可得到正常结果,除非有实际的元素名使用这两个名称。

\subsection{私有方法控制}

为确保完全与对象数组一致,隐藏细节。将以下一些方法设为私有。

set.child
set.parent
parentas
parentto

\subsection{标量方法}

以下方法仅设计为只能由标量对象调用,不能由对象数组调用。

get.atnmae
get.vector
parentas
rootas
pushChild
popChild
deleteChild
indexChild
nextSibling
prevSibling

\subsection{xpath}

xml 与 xpath 是密不可分的。要实现 xpath 筛选满足条件的元素结点,即是 xmlel 对
象数组。不过对由于元素属性不是对象,用 xpath 提取属性会有点不同表现。

函数式使用 xpath 有 query 与 queryset 两个方法,后者是赋值方法。

node = query(obj, subpath, atnames)
obj = queryset(obj, subpath, atnames, val)

subpath 支持最简单的 xpath 语法。路径中最后的属性结点部分可独立出来用 atnames
参数输入。atnames 是 cellstr 类型,支持查询多个属性。属性结点也可以 @name 形式
放在 subpath 参数的最后一部分,按 xpath 的语法只允许指定一个属性名,不过 query
方法也允许指定多个属性名,如 /path/to/element/@name1,@name2。

如果在 subpath 或 atnames 中指定了属性,query 返回属性值数组,cellstr。否则反
返回 xmlel 对象列表。在 queryset 的输入新值 val 的大小,必须与 query 查询出的
属性值数组大小一致,省略 atnames 或留空表示查询所有属性。

也重载了点索引,subsref 与 subsasgn 。这样就可以用点号渐次索引子元素,代替将
xpath 字符串传给 query 方法。但依 matlab 自身语法,这种索引方式不能直接嵌入中
括号条件,也不能使用 @ 引用属性名,只能混合点索引子元素以及小括号整数索引特定
位置的子元素。重载索引用递归实现,query 用循环实现。建议只在命令窗中使用快捷的
索引方式,而在脚本中使用 query 方法。

辅助方法,filterat,用于解析 xpath 的条件筛选,目前只支持按属性筛选,即在中括
号无法使用太重复的嵌入 xpath 语法。getChild 可根据整数索引或元素名返回子元素,
没有相应名称的子元素,则尝试返回相应名称的属性值,也可用 @ 前缀限定返回属性值
。

传入 query 的 xpath 字符串,如果以 / 开始,认为是绝对路径,否则是相对路径。相
相对路径从当前元素(调用者)的子元素向下索引。绝对路径时,首先要调用 getRoot
方法上溯找到自己所依赖的根元素(找不到时 query 返回空)。按 xpath 语法,/ 之后
应该接上根元素的元素名。在 query 方法中,为简便起见或考虑到经常容易遗漏根元素
,也支持省略根元素名,直接在 / 后开始索引根元素的第一层子元素。

query(xpath) 或重载的点号索引,中间元素都可以是对象数组,表示对数组的所有元素
继续向下索引再合并。注意 matlab 原生的 struct 域索引,对 struct 数组索引无法连
续的,一般要求结标量索引域。

\subsection{静态方法}

一些与类方法密切相关的工具函数,打包为该类的静态方法,使之更具有独立性。

obj = cast(var, name, atname)
obj = createTextNode(str)
cpath = pathseg(pathstr, onlyname)
tf = isscalarval(val)
c = tostrval(d, keep)
d = tonumval(c)

cast 用于将常规变量转为 xmlel 对象。createTextNode 用于创建一个文本元素。
pathseg 用于辅助解析 xpath,将 xpath 分解为几部分,存于 cell 数组中。xml 每个
属性只能保存标量,故用 isscalarval 来判断是否标量数字或字符串。tostrval 与
tonumval 在数字矩阵与字符串矩阵(cellstr)中相互转换。

\section{外部支持需求}

一些方法的实现要调用其他一些方法,考虑到比较通用,放在了 utility 工具箱下。
另外一些更相关的小函数,已复制为类的静态方法。

tomat 方法要用到 iscellstruct 判断

解析 xmlfile 源文件,用 xmlfile 类,parseFile('xmlel')

\section{Bugs}

1. atMatrix 在将属性值转为矩阵是,要求同名元素的属性名个数与顺序一致。

2. 退出 matlab 自动保存的对象,重新打开 matlab 自动载入对象会有问题。但是在同一个
会话中,用 save load 却没问题,不知为何。

3. 数组对象不能点号索引属性,必需加括号索引当作方法调用。不算重载索引的 bug ,
而是 matlab 本身的规范。比较清晰的坐法是索引方法始终不要省略括号。

\section{Todos}

\section{更新说明}

由于在 xmlel 对数数组上重载索引有些小问题。故将 xmlel 类的重载索引取消。另建一
个包装类 xmldoc 类,实现索引。xmldoc 维护一个根元素,及 xpath 路径。xmldoc 对
对象可以连续索引子元素,找不到子元素时尝试调用内含的 xmlel 方法。除了索引外,
也实现 query 与 queryset 接口,可以使用默认的保存在对象的 xpath。

\end{document}
